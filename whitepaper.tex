\documentclass{article}
\title{FinneyVote: An Incentivized Public Forum}
\author{William Morriss}

\begin{document}
\maketitle
\begin{abstract}
FinneyVote is the first decentralized forum designed to promote controversy.
Voters are certified by a small refundable deposit.
Accounts vote with cases, rewarding casemakers.
Proposers are rewarded per-case, incentivizing controversy.
A decentralized autonomous organization (DAO) profiting from vote token sales governs and supports the system.
\end{abstract}
\section{Necessity}
With the upcoming advent of strong artificial intelligence, it is imperative to establish a decentralized and auditable public forum.
Scalable intelligence must participate in our discussions and confront our ideas, else we may diverge catastrophically.
\par
Before then, decentralized voting offers promising secured political expression.
Censorship runs rampant in authoritarian regimes.
China supports a Great Firewall and suppresses anti-regime dialogue.
Even in western countries, social media websites such as Facebook censor unpopular ideas.
No centralized service can be trusted with voting because it can manipulate the result.
Political expression should be trustless and uncensorable.
\par
Democracies give political voice to commoners.
However, commoners may not be sufficiently educated on, or responsible for, the matters they decide.
Republics solve this with representation, where commoners choose delegates to vote on their behalf.
Unfortunatley, representatives are corruptible.
Special interests lobby them to make unpopular decisions that will not necessarily cost them re-election.
\par
Further, delegate selection often disenfranchises minority opinions.
Geographic representation, implemented as districts, leads to gerrymandering.
Many representation systems are first-past-the-post, which leads to a two-party system, where two generic parties compete for independent voters.
Since the parties are not based in ideology, voters are often torn between mundane representatives that do not fully represent themselves.
Proportional representation more-accurately represents political demographics, but still categorizes unrelated positions into ideological parties.
Opinions lacking adhesion to an electable ideology remain unrepresented.
Even if the majority support a policy, if it's not endorsed by the majority coalition it cannot take effect.
\par
FinneyVote supercedes politicians by replacing them with ideas.
When taking a position on a proposal, voters choose an argument to represent their viewpoint.
An uninformed voter can review the discourse and make a reasonable choice aligned with their values.
\section{Accounts, Identities, and Cabals}
Ethereum provides an incentivized decentralized network simulating a fully programmable computer.
Accounts are identified by their public key and controlled by their private key.
The ease of creating new accounts obliges Sybil mitigation for open voting systems.
FinneyVoters are certified by a deposit of one milliether, canonically called a finney, after Hal Finney.
\par
Registered accounts earn an allowance of 2 votes per day.
A registered account can only deregister and reclaim their finney after a week.
Individuals interested in augmenting proposal vote counts in favor of their ideas will stake many accounts.
Such power users will amass ``powerbanks'' of accounts to cast more votes per proposal.
The aggregate voting power of powerbanks is limited by their stake in finney.
Because of the transaction fees of powerbanking, a system of accounts that vote together is a unique identity.
The number of votes required to establish your identity is logarithmic in the population size, assuming the proposal system properly incentivizes controversy and variety of topic.
\par
Powerbanks profit the DAO; she will never disallow them.
To overcome powerbank sybil, groups of verified unique identities will organize into ``cabals''.
Cabals are DAOs of like-minded individuals.
Cabals can provide their members a subscription of proposals for curation.
They can expose an interface for counting the votes of all their members on proposals.
Cabals can determine canon, a set of proposals accepted by all of their members.
Cabals can govern pooled powerbanks.
Cabals can define ideologies and facilitate panarchy.
\section{Incentives}
Voters give half their vote to casemakers.
Casemakers give half their vote to proposers.
Proposals are incentivized to attract many cases.
Cases are incentivized to attract many votes.
Creating a new case is gassier than voting with an existing one, and similar cases compete for similar voters, so cases differ to represent portions of the anticipated voting population.
The continuous faucet encourages voters to return daily and check new proposals, which compete for their attention.
\par
If the entire vote was transfered as incentive, powerbanks would be able to vote for their own arguments to exercise influence quadratic in their stake.
When only half of a vote is awarded, votes can only be used twice.
The governing DAO collects the difference and profits by selling its excess on an open market, such as EtherDelta.
Thus, in order to be profitable, the DAO must promote aggregate voting.
\par
If accounts were free to vote on every proposal, 
\par
Third-party mining of FinneyVotes for sale is unlikely to be profitable given the comparative advantage of the owner account, who does not have to mine with gas and staked finney.
The ability to mine with gas and staked finney creates an elastic supply curve.
Critically, the expected value of a FinneyVote should decrease over time with the gas price.
With an inflationary token, voters are incentivized to use their vote.
\par
Currently, the token is governed by William Morriss directly.
Ownership will be transfered to the DAO pending the release of Aragon.
The DAO will then ICO.
\section{Upgrades}
As with CryptoKitties and Proof of Weak Hands, there may be many copycats offering a similar service.
The ability of FinneyVote to release updates will limit its ability to compete.
While the user interface can be updated trivially, the smart contracts are more permanent.
\par
In the worst case, the token can migrate account registries.
Users can deregister and move to the new registry.
This approach is not preferrable because of the massive transaction fees required of the users.
\par
The first account registry allows trusted parties to create proposals.
Thus, the entire proposal system can be arbitrarily upgraded without migrating account registries.
\par
Some features can be implemented as separate contracts, as some front-ends will choose not to render the feature.
For example, comments can be added as stand-alone contracts without requiring an update to proposals.
By preferring separate contracts, the system remains modular, and clients can experiment with new features without forking the community.
\section{FinneyVoters as a Service}
Until verified unique identity can be cheaply established, many services will also use time-staked deposits for anti-Sybil.
The account registry can be used to validate FinneyVoters in such contexts.
Third-party polls, for example, can mitigate spam by verifying their participants are registered FinneyVoters.
It is much better for users to have one deposit for all services than for each to require their own.
The account registry can also approve outside proposals to use the FinneyVote token.
Such proposals could raise funds, take action, or determine goals democratically.
\section{Related Work}
The SomethingAwful Forums require users to pay a one-time registration fee of $\$9.95$.
This fee mitigates spam while profiting the service.
FinneyVote does not pocket the registration deposit and instead allows accounts to deregister and reclaim their deposit.
This maximizes aggregate voting by reducing the perceived cost of account creation to zero.
In practice, some accounts will never deregister, even after account registry migration.
\par
Reddit allows its users to upvote or downvote posts and comments, creating a crowdsourced content curation service, the front-page of the internet.
Steem operates a similar but decentralized system that rewards content providers with cryptocurrency.
Payouts are slowed by a vesting schedule.
Steem boasts a feeless network with more transactions than Ethereum.
Similar to FinneyVote, Steem requires a minimum balance of STEEM to participate and rewards its contributors.
While both are effective content curation systems, neither are effective forums for controversy.
Unpopular content is buried and political communities circle-jerk.
Because Steem is feeless and rewards voters, Steemit struggles with sybil promotion services, which are weakening the quality of its content.
The most powerful Steemit accounts exert tens of thousands times more power than the least.
Steem wants to keep upvoting feeless so that users do not hesitate to upvote.
Instead, FinneyVote wraps political voting with opportunity costs, to perturb powerbanking and promote representative discussion.
\par
Aragon is developing smart contracts for the governance of DAOs.
They support a number of voting systems, including voting weighted by share.
Because members do not have to divide their influence among many accounts, weighted voting is cheaper than finneyvoting in terms of transaction fees.
Instead, powerbanks pay transaction fees proportional to their deposit.
Influence must be linear both in deposit and fee, else powerbanks have a cost-advantage over groups of single-account voters.
\par
Transferring registration deposits to a DAO would give FinneyVote arms and legs.
Some proposals could take action with the funds if they have enough votes after a voting period.
Dissenters opposed to such actions would not want their funds to support it, and the community could fracture.
DAOs already provide this functionality more-effectively, and are better-equipped to take action.
In general, action is best governed by responsible stakeholders, or perhaps futures markets.
\par
Liquid democracy is a proposed democratic system that would allow voters to recursively delegate their votes to others.
FinneyVote allows this through the creation of smart-contract voters, who allow their owners to delegate their votes on specified proposals to trusted delegates.
More likely though, voters would simply vote with arguments written by authoritative casemakers.
Cases are immutable, while delegates can be fickle.
FinneyVote provides liquid democracy implicitly when voters choose existing cases.
\par
Futarchy is a proposed system of government that elects leaders to choose community goals, but uses futures markets to decide how to achieve them.
FinneyVote can supplant elected representatives for determining value-priorities.
\end{document}
