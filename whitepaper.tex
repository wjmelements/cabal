\documentclass{article}
\title{FinneyVote: An Incentivized Public Forum}
\author{William Morriss}

\begin{document}
\maketitle
\begin{abstract}
FinneyVote is a decentralized forum where registered accounts vote on proposals.
Accounts vote with cases, rewarding casemakers.
Casemakers in turn pay proposers, incentivizing controversy.
A decentralized autonomous organization (DAO), supported by vote token sales, governs and supports the system.
A sufficiently unique voting history can establish unique identity.
Sybil voting is impaired by the registration staking requirement and the voluntary organization of unique-user groups.
\end{abstract}
\section{Necessity}
With the upcoming advent of strong artificial intelligence, it is imperative to establish a decentralized and auditable public forum.
\par
Democracies give political voice to commoners.
However, commoners may not be sufficiently educated on the matters they decide.
Republics solve this with representation, where commoners choose delegates to vote on their behalf.
Unfortunatley, representatives are corruptible.
Special interests lobby them to make unpopular decisions that will not necessarily cost them re-election.
\par
Further, delegate selection often disenfranchises minority opinions.
Geographic representation, implemented as districts, leads to gerrymandering.
Most representation systems are first-past-the-post, which leads to a two-party system, where two generic parties compete for independent voters.
Since the parties are not based in ideology, voters are often torn between mundane representatives that do not fully represent themselves.
Proportional representation more-accurately represents political demographics, but still categorizes unrelated positions into ideological parties.
Opinions lacking adhesion to an electable ideology remain unrepresented.
Even if the majority support a policy, if it's not endoresed by the majority coalition it cannot take effect.
\par
FinneyVote supercedes politicians by replacing them with ideas.
When taking a position on a proposal, voters choose an argument to represent their viewpoint.
An uninformed voter can review the discourse and make a reasonable choice aligned with their values.
\section{Accounts, Identities, and Cabals}
Ethereum provides an incentivized decentralized network simulating a fully programmable computer.
Accounts are identified by their public key and controlled by their private key.
Anyone can generate a keypair, so when voting in Ethereum it is necessary to prove an account is a unique individual.
\par
Accounts require a deposit of 1 finney and earn an allowance of 2 votes per day.
A registered account can only deregister and reclaim their finney after a week.
Individuals interested in augmenting proposal vote counts in favor of their ideas will stake many accounts.
Such power users will amass ``powerbanks'' of accounts to cast more votes per proposal.
The aggregate voting power of powerbanks is limited by their stake in finney.
Because of the transaction fees of powerbanking, a system of accounts that vote together is a unique identity.
The number of votes required to establish your identity is logarithmic in the population size, assuming the proposal system properly incentivizes controversy and variety of topic.
\par
To overcome powerbank sybil, groups of verified unique identities will organize into ``cabals''.
Cabals are DAOs of like-minded individuals.
Cabals can provide their members a subscription of proposals for curation.
They can expose an interface for counting the votes of all their members on proposals.
Cabals can determine canon, a set of proposals accepted by all of their members.
Cabals can define ideologies and facilitate panarchy.
\section{Incentives}
Voters give half their vote to casemakers.
Casemakers give half their vote to proposers.
Proposals are incentivized to attract many cases.
Cases are incentivized to attract many votes.
The continuous faucet encourages users to return daily and check new proposals, which compete for their attention.
\par
If the entire vote was transfered as incentive, powerbanks would be able to vote for their own arguments to exercise influence quadratic in their stake.
When only half of a vote is awarded, votes can only be used twice.
The governing DAO collects the difference and profits by selling its excess on an open market, such as EtherDelta.
Thus, in order to be profitable, the DAO must promote aggregate voting.
\par
Third party mining of FinneyVotes for sale is unlikely to be profitable given the comparative advantage of the owner account, who does not have to mine with gas and staked finney.
The ability to mine with gas and staked finney creates an elastic supply curve.
Critically, the expected value of a FinneyVote should decrease over time with the gas price.
With an inflationary token, voters are incentivized to use their vote.
\par
Currently, the token is governed by William Morriss directly.
The ownership will be transfered to the DAO pending the release of Aragon.
The DAO will then ICO.
\end{document}
